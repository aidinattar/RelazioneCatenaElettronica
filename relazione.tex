\documentclass{article}  
\usepackage{graphicx}
\usepackage[utf8]{inputenc}
\usepackage[T1]{fontenc}
\usepackage{float}
\usepackage[italian]{babel}
\usepackage{listings}
\usepackage[usenames]{color}
\usepackage{natbib}
\usepackage{siunitx}
\usepackage[strict]{changepage}
\usepackage{physics}
\usepackage{wrapfig}
\usepackage[a4paper, top=2cm, bottom=2cm, right=2cm, left=2cm]{geometry}
\usepackage{array}
\usepackage{color}
\usepackage{colortbl}
\usepackage{amsmath}
\usepackage{amssymb}
\usepackage{multirow}
\usepackage{enumitem}
\usepackage{hyperref}
\usepackage{times}
\usepackage{booktabs}



\title{Relazione di laboratorio: studio di catena elettronica}
\author{Docente: dott. Garfagnini, dott. Lunardon \\
Gruppo 14}
\date{Anno accademico 2019/20}

\begin{document}



\maketitle

\begin{itemize}
    \item[$\circ$] Aidin Attar - 1170698 - aidin.attar@studenti.unipd.it
    \item[$\circ$] Ema Baci - 1171107 – ema.baci@studenti.unipd.it
    \item[$\circ$] Alessandro Bianchetti – 1162147 – alessandro.bianchetti@studenti.unipd.it
\end{itemize}

\vspace{3 cm}
\begin{large}\textsc{\textbf{Scopo dell'esperienza}: studio della curva di trasferimento di un amplificatore invertente, del circuito con amplificatore delle differenze (non invertente), dei circuiti sommatore invertente, derivatore, raddrizzatore di precisione.} 
\end{large}
\vspace{8.5cm}

\begin{figure}[H]
\centering
\includegraphics[scale=0.5, angle=0]{unipd_logo.png}
\end{figure}

%\newpage \tableofcontents \newpage

\twocolumn



\section{Conclusioni}


\appendix
\section{Appendici}
\label{appendice}
\subsection{Costruzione dell'errore sulle misure}
\label{Calcerr}

Nel trattare i dati rilevati dall'oscilloscopio nel corso dell'esperienza si sono assegnati gli errori alle misure tenendo conto che ogni misura è affetta da un'incertezza di origine sistematica e da una di lettura, dovuta al posizionamento dei cursori sulla schermata. Per semplificare il calcolo degli errori, si sceglie di considerare un errore massimo $\Delta_{\%}$ di tipo percentuale per identificare il contributo sistematico presente nella presa dati, e un errore massimo $\Delta_{lett}$ per coprire le fluttuazioni casuali. La percentuale del valore letto da utilizzare come $\Delta_{\%}$ è del $3\%$ per le tensioni e dello $0.01\%$ sui tempi: per quanto riguarda invece l'errore di lettura, si è considerato 1/10 della divisione utilizzata.
Tuttavia, poiché l'errore percentuale sui tempi è sempre decisamente trascurabile rispetto a quello di lettura, lo si è omesso nel calcolo dell'incertezza totale.

Infine, si precisa che tutti i risultati sono presentati con un errore non massimo, ma di tipo statistico: si riporta la regola di conversione, in ipotesi di distribuzione uniforme per l'errore sistematico e in ipotesi di distribuzione triangolare per l'errore di lettura.

\begin{equation}
\sigma_{\%}=\frac{2\Delta_{\%}}{\sqrt{12}} \quad \quad \sigma_{lett}=\frac{2\Delta_{lett}}{\sqrt{24}}
\end{equation}

Un meccanismo analogo vale per le misure dirette di grandezze quali resistenze e capacità effettuate con il multimetro: anche in questo caso abbiamo un contributo sistematico e uno casuale, il primo dato ancora da un errore percentuale e il secondo da un errore in digit. Tale contributo in digit è dato da $\Delta_{dgt}=ns$ dove $n=\#digit$ è un numero intero riportato sul manuale dello strumento e $s$ è la sensibilità usata nella lettura del valore. Anche qui si utilizza la conversione in errori statistici in ipotesi di distribuzione uniforme.

\subsection{Commento sull'accettazione/rifiuto dei fit}
Nel corso della relazione sono stati riportati diverse volte i parametri per la verifica della bontà del fit, sostenendo che essi permettessero di accettare o rifiutare l'interpolazione.


Per quanto riguarda i valori di $\chi^{(2)}$, l'ipotesi che il fit descriva i dati viene accettata o rifiutata con 0.90 CL, mentre
per il valore di $t$ di Student l'ipotesi di non correlazione è stata rifiutata o accettata con lo stesso CL.


\subsection{Tabella delle compatibilità}
\medskip
\begin{table}[H]
    \centering
    \begin{tabular}{c}
        %\hline
        \begin{Large}
        $\lambda=\frac{|a-b|}{\sqrt{\sigma_a^2+\sigma_b^2}}$
        \end{Large}\\
        %\hline
    \end{tabular}
    \hspace{0.5cm}
    \begin{tabular}{cc}
        \toprule
        &       \textbf{Compatibilità   }       \\
        \midrule
        0$\leq \lambda$<1   &Ottima                 \\
        1<$\leq \lambda$<2   &Buona                  \\
        2<$\leq \lambda$<3   &Accettabile            \\
        3<$\leq\lambda$<5   &Pessima                \\
        $ \lambda \geq $  5     &Non compatibile        \\
        \bottomrule
    \end{tabular}
    \caption{Indicazioni lettura compatibilità}
\end{table}

\subsection{Dati sperimentali}
\end{document}
